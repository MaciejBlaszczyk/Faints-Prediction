\chapter{Introduction}


\quad Nowadays, research and development of artificial intelligence has become widespread in every field of science. The emerging trend of completely replacing man with a machine poses new challenges to create a human-like being in the context of decision-making skills. The human ability to perceive the world (the environment in which it operates) and the ability to learn is a major challenge when trying to create beings similar to humans. 
Joint decision-making might be better than in the case of a decision by a single individual. A system as a structure composed of agents who have the ability to solve tasks based on decision-making. In this work, various approaches to agent learning process will be considered to create final prediction. By appointing agents-experts in a given field by modifying their attributes of which they perceive the environment, there is a chance to get a better prediction model.

In the bibliography, there will be modern state-of-the-art papers described multi-agent systems, artificial intelligence used in these types of systems and some most important milestones of this field of knowledge as well as some psychological works about human behaviour.

\section{Goal of the thesis}


\quad The goal of this work is to research and analyze the strategies of learning in agent systems. After analyzing actual solutions which are created over multi agent systems would be considered style of learning and perceving environemnt by each agent.
In these work will be used advanced ensemble learning techniques. Different types of approaches will be used to construct the final prediction model. What differentiates this work from other projects, is mainly the approach to learning for each agent. An environmental overhead that limits the perception of agents, making them specific experts in a given field
As part of the work will be the creation of simulation groups having human traits whose task will be to take the best decision. Strategies of using agents in service are discussed, in relation to the imposed conditions by the system environment. The creation of the model by the agent with limited ability to perceive the environment will be checked for many combinations of available machine learning algorithms. What's more, combining these models, including those that proved to be relatively inferior, will improve the final predictive model.
Finally, the problem under consideration will be a set of data determining the prices of plots (including houses) in the USA based on parameters determining: geographical location, size of the property, condition and construction of the house.

\section{Elo}

\quad siema elo
co tam
