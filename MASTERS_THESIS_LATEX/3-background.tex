\chapter{Background}
% W tym rozdziale przedstawiona jest architektura modeli deep learningowych wykorzystywanych w tej pracy. Najpierw opisane sa podstawowe jednostki z  ktorych skladaja sie sieci neuronowe, a nastepnie wyjasnione sa cale architektury uzytych modeli.
This chapter presents the theoretical background behind the architecture of deep learning models used in this work. First of all the basic units that compose the neural network are described. Then there is an explanation of the whole architecture of the chosen deep learning models.


\section{Neural Network Layers}
% Istnieje duzo typow warstw sieci neuronowych, z ktorych kazdy wykonuje rozne operacje matematyczne, co pozwala mu specjalizowac sie w konkretnych zadaniach. Podczas budowania sieci neuronowej mozna stackowac na sobie rozne warstwy, co pozwala na zbudowanie nieskonczonej liczby roznych modeli. W tym podrozdziale opisane sa wybrane warstwy sieci neuronowych, ktore wykorzystano w tej pracy.
There are many types of neural networks' layers, each of which performs different mathematical operations, which allows it to specialize in specific tasks. When building a neural network, you can stack different layers on each other, which allows you to build an infinite number of different models. This section describes selected layers that were used in this work to build the models .


\subsection{Dense}
Dense layers are super easy pro layers


\subsection{Convolution}
Convolution layers catch local patterns

\subsection{Long-Short Term Memory}
LSTM layers are better than recurrent layers

\subsection{Attention}
Attention layer helps to focus on proper features


\section{Neural Network Architectures}
% Aby dobrze dzialac i skutecznie wykonywac wyspecyfikowane zadanie, model wymaga odpowiednio dobranej architektury. Co roku powstaja kolejne coraz bardziej lub mniej skompikowane schematy wg. ktorych nalezy budowac modele glebokiego uczenia tak aby osiagnac satysfakcjonujace rezultaty. W tym podrozdziale opisane sa schematy architektur sieci neuronowych wykorzystanych w tej pracy.
To successfully perform a specified task, the model requires a properly selected architecture. Every year, more and more complicated schemes are created in order to solve more sophisticated tasks. Usually to obtain satisfactory results, a well known scheme should be followed mhile building a model. This subsection describes the schemes of neural network architectures used in this work.


\subsection{AutoEncoder}
AutoEncoder is unsupervised network.

\subsection{Variational AutoEncoder}
VAE is generative network.
